\documentclass{article}
\usepackage[utf8]{inputenc}

\usepackage{tabularx, booktabs}

\usepackage{amsmath, amssymb}
\usepackage[margin=1in]{geometry}
\usepackage{graphicx}
\usepackage{xcolor}

% To set the enumeration method
\usepackage[shortlabels]{enumitem}

\usepackage[T1]{fontenc}
\usepackage[utf8]{inputenc}
\usepackage{lmodern}

\title{PM511 b: Anthony work Assignment 4}
\author{Group 7}
\date{Due: Feb 8, 2019}

\begin{document}

% We use this command to change the font of the entire document. We can take it out if you want :)
\fontfamily{lmss}\selectfont

\maketitle

\section{question 1}
 Evaluate the association between the dependent variable of high systolic blood pressure (cut at 140 mmHg) and independent variables of age (as a continuous variable; center the age variable at age 60) and overweight.
 \begin{itemize}
     \item a:  $\textbf{Evaluate each independent variable (age and overweight) in a separate model}$.  Using logistic regression of high-blood pressure ($>140mmHg$) on age, the predicted logit is -1.51 + 0.0828*(Age). For person aged 60, the predicted probability of high-blood pressure at the age of 60 is 0.181.   The odds of having high-blood pressure is 1.087 times higher per year increase of age (1.073,1.099).  Age is statistically associated with high-blood pressure (LRT=188.54, p$<0.0001)$.
     
      The logistic regression model of high-blood pressure on overweight indicates that overweight is statistically associated with HBP (LRT=17.15,df=1 p$<0.001$).  The odds ratio of HBP of someone who is overweight is 1.65 (1.29,2.11) times as likely than someone who is not overweight.
     \begin{verbatim}
     STATA CODE
  *question 1a
  glm hbp centage, family(binomial) link(logit)
  logit hbp centage
  ***overweight
  glm hbp overweight, family(binomial) link(logit)	
  logit hbp overweight
     \end{verbatim}
     \item b:  Evaluate the two independent variables in the same model.
     Using multivariate logistic regression of HBP on age (centered on 60), and overweight, the model was significant indicating there is a significant association between age, overweight and HBP (LRT chi-square =217.84, $p<0.001$).  Age centered was a significant explanatory variable (Wald's p$<0.001$, z=13.10), and for 1 year increase in age, the estimated logit increases 0.087 (0.074,0.0995).  Overweight was a significant explanatory variable (Wald's p$<0.001)$, Z=5.21), and for change in overweight status, the predicted logit increases 0.688 (0.43,0.95).  The predicted logit for someone aged 60, and not overweight is -2.02, (estimated probability : 0.133)
    STATA CODE:
     \begin{verbatim}
          *question 1b multiple LR
          logit hbp centage overweight
          glm hbp centage overweight, family(binomial) link(logit)
     \end{verbatim}
     
     \item c:  Create a table of results that reports the unadjusted and mutually adjusted parameter estimates, along with 95$\%$ confidence intervals. 
     
       \begin{table}
          \caption{\label{tab:table-name} logit estimates}
\begin{tabular}{llll}
 & Unadjusted (univariate model)  & Adjusted  & absolute Percent adjusted  \\
age (Age-60) & 0.083 (0.0703,0.0954)  & 0.087 (0.0734,0.0955) & 4.83$\%$   \\
 overweight &  0.501 (0.256,0.745) & 0.688 (0.429,0.946)  & $37.33\%$  \\
\end{tabular}
\end{table}
     
     
     \item ci:  Is there any evidence to suggest that the association of overweight with high blood pressure is confounded by age?  Explain your answer.
      Age and overweight are both significantly associated with high-blood pressure.  In order to examine confounding, we performed a logistic regression of overweight (response) on age (independent predictor).  This model indicated that age and overweight status were significantly associated (LRT chi-square =10.56, $p<0.0012$, DF=1).  We conclude that age is a confounder in the relationship between high-blood pressure and overweight status because we have the 3 way relationship and age adjusts the overweight status estimates by $37.33\%$.
      \begin{verbatim}
           *question 1c, 
           di 100*(0.5-0.688)/0.5
           *37.8%
           *3 way relationship
       logit overweight centage 
	   reg centage overweight
      \end{verbatim}
     
     \item d: Write a short paragraph interpreting your parameter estimates and providing a basic conclusion regarding the association of high blood pressure with age and overweight.
      Using multivariate regression of high-blood pressure on age (centered on 60) and overweight status, our model indicated that at least one of the independent variables were significantly associated with HBP (LRT Chi = 217.8,p$<0.001)$.  The adjusted predicted logit increases 0.087 (0.0734,0.955) per 1 year increase in age.  The adjusted predicted logit increases 0.688 (0.429,0.946) per change in overweight status. The predicted logit (log-odds) for a 60 year old person, not overweight is -2.02.
      The adjusted predicted odds ratio increases 1.09 (1.077,1.10) per 1 year increase in age.  The adjusted predicted odds ratio increases 1.99 (1.54,2.58) per change in overweight status. 
      
      The relationship between HBP and overweight was confounded by age, and we observed that age was significantly associated with overweight status upon examination of simple logistic regression of overweight on age.  
      
      The linear predictor of high-blood pressure on the logit scale is defined as :$-2.02+ 0.688*Overweight +0.087*Age$
      the predicted probability of high-blood pressure $\hat{\pi}$ = exp($-2.02+ 0.688*Overweight +0.087*Age$)/(1+exp($-2.02+ 0.688*Overweight +0.087*Age$))
      
     \item e:  Using the fitted two-variable model (age and overweight), calculate the logit and the predicted probability of high blood pressure among: a person aged 70 who is overweight. A person aged 50 who is not overweight.
     
     For someone aged 70 and overweight, the predicted logit of HBP is -0.464 (-0.626,-0.302), and the predicted probability of high blood pressure is 0.386. 
     
      For someone aged 50 and not overweight, the predicted logit of HBP is -2.886 (-3.18,-2.59), and the predicted probability of high blood pressure is 0.053 (0.083,0.15) 
      \begin{verbatim}
      *question 1e, predicting probabilities and logit from the Multiple regression.
      *E(Y)= -2.019 + 0.0867*centage+0.68759*overweight
      *person is is 70 and overweight.
      di -2.019 + 0.0867*(70-60)+0.68759*
      *the predicted logit is -0.46441
      di exp(-2.019 + 0.0867*(70-60)+0.68759*1)/(1+exp(-2.019 + 0.0867*(70-60)+0.68759*1))
      *the predicted probability of hbp is 0.386
	   *person 50 years not overweight
    di -2.019 + 0.0867*(50-60)+0.68759*0
	di exp(-2.019 + 0.0867*(50-60)+0.68759*0)/(1+exp(-2.019 + 0.0867*(50-60)+0.68759*0))
	*the predicted logit is -2.886
	*the predicted probability is 0.053
      \end{verbatim}
 \end{itemize}


\section{Question 2} 
  Explore the relationship between high blood pressure and age more carefully.  Specifically, is the relationship linear on the logit scale?  Evaluate the relationship using:
\begin{itemize}
    \item a:  Grouped smooth with quartiles.  The grouped smooth twoway plot indicates a linear relationship on the logit scale between high blood pressure and age. 
    \begin{verbatim}
      *first cut into quantiles
	  summarize AGE,detail
	egen AGEcat=cut(AGE),at(30,53,59,65,90) icodes
  *check categories
   tabstat AGE,by(AGEcat) stats(mean min max)
  bysort AGEcat: egen midage=mean(AGE)
   list AGE AGEcat midage in 1/5
   
   *i.AGEcat creates indicator variables for each factor
    logit hbp i.AGEcat
	
	gen beta=0
	replace beta=0.9408238 if AGEcat==1
	replace beta=1.4633 if AGEcat==2
	replace beta=2.121252 if AGEcat==3
	
	twoway (line beta midage) (scatter beta midage)
    \end{verbatim}
    \item b: Lowess smoothing: The lowess plot shows a dip, and looks semi-linear.  Upon looking at the dot plot overlay, we can see there are outliers and the data appears skewed because of a few very young subjects.
     \begin{verbatim}
         *2b LOESS smoothing
		lowess hbp centage, logit
    	lowess hbp centage, addplot(scatter hbp centage) legend(off)
	     \end{verbatim}
    
    
    \item c: Fractional polynomials.   What is the best one-term and the best two-term model?   Are these models better than the linear model (give p-values)? 
    The linear term is not significantly different from the 2-term model (p=0.11), neither is the 1-term model (p=0.22).  The linear term is not different from the 1-term model (p=0.083).
     \begin{verbatim}
         fp <AGE>, dimension(2) replace: logit hbp <AGE>
	*the linear term p-value =0.111 does not differ from 2-term
	
	*1-term model.
	fp <AGE>, dimension(1) replace: logit hbp <AGE>
 *the 1-term is -0.5, b0+-79.94*(1/sqrt(AGE)), and the linear model is not significantly
 *different p=0.083.
	
	*2-term model
	*b0 + 64827.96*(1/AGE)^2 + -20750.29*(1/AGE)^2*Ln(AGE)
*the linear model is not different from 2-term, p=0.22
	
     \end{verbatim}
    
    \item d: Based on all of your explorations, how would you choose to model the relationship between high blood pressure and age?  Explain your answer, using the information you obtained from each of the 3 approaches (2a-2c).  
     The 1-term model has the power =-0.5, and the equation is b0-79.94*(1/$\sqrt{AGE}$) and the linear model does not differ (p=0.083).
     The 2-term model has power -2,-2, and the equation is b0+64827.96*($1/AGE)^{2} - 20750.29*(1/AGE)^{2}Ln(AGE)$.  We will choose the best linear predictor equation because it does not differ from the power terms, and is most interpret-able. The logistic linear equation for age centered at 60 can be used (logit(p) = -1.51+0.082*AGE.  You can also use indicator variables for age quartiles which is defined as logit(p) = -2.76+ 0.94*$AgeQ2$+ 1.46*AgeQ3 + 2.12*AgeQ4
    
\end{itemize}

\section{Question 3} 
Now use the logistic regression model above to model age in two ways: (1) mean-centered age as a continuous variable; (2) Age categories $(<45, 45-49, 50-54, 55-59, 60-64, 65-69, 70-74, 75-79, and \geq 80). $
Run each age variable in a model with the overweight variable.
 After dropping missing variables for age (continuous), overweight, HBP, we have 2325 observations.  
\begin{itemize}
    \item a: Obtain the Wald’s test and likelihood ratio test for age in each model.(Hint: the df for the continuous age = 1; the df for the categorical age variable = 8).
        For age mean centered at 58.87, the likelihood ratio test of a reduced model using HBP$\sim overweight$ and full model $HBP\sim meanAGE+ overweight$ has LRT statistic of 200.72 with 1 DF, and is significant. hence we conclude that age is an important variable in this relationship.
        
        
        The Wald test on meanAGE had a Chi-square test statistic of 176.88, 1 DF and was significant.  We conclude that the association of high-blood pressure and overweight does differ by age, age is an important explanatory variable. The multivariate wald test on the full model had Chi-square of 187.98, with $p<0.001$ indicating at least one of the explanatory variables is not equal to zero.
        Upon investigation of interaction between mean age and overweight,, there was not significant interaction effect modification of overweight and mean age (p=0.893).
        \begin{verbatim}
       * 1) mean centered age as continuous
	 summarize AGE,detail
	  generate meanAGE=AGE-58.87398
	  logit hbp meanAGE overweight
	   summarize meanAGE,detail
	   
	   *model 1) 
	* drop if missing(meanAGE) | missing(overweight)
	 	 logit hbp meanAGE overweight
	    logit hbp overweight##c.meanAGE	
	
 *likelihood ratio test for each variable
   	 logit hbp meanAGE overweight
	 est store full
	 logit hbp overweight
	 est store reduced
	 lrtest reduced full
	 *LRT (L(reduced=-1102.43 - L(full)=-1005.87) = 193.131
	 
 *wald test for each variable
   logit hbp meanAGE overweight
   test meanAGE
 \end{verbatim}
        Now examining the 8 categorical variables of age, we fit a logistic model without interactions between overweight status and age categories, and compare to a model that does not include age.  The Liklelihood difference is 202.67, p<$0.001$, DF=8. This indicates that the 8 age categories are a significant variable in the association between hbp and overweight status.
           The wald test of the 8 categories (Chi-square=168.78, DF=8, $p<0.001$) for each age category is statistically significantly different from 0, hence we reject the null that all age categories are equal to 0, and there are at least 1 age category with logit estimates which differ from 0.
      
        \begin{verbatim}
**categorical variable analysis
	egen newAGEcat=cut(AGE),at(22,45,50,55,60,65,70,75,80,100) icodes
  	 logit hbp i.newAGEcat overweight
	 est store full
	 logit hbp overweight
	 est store reduced
	 lrtest reduced full
		 
	 logit hbp i.newAGEcat overweight
	 
 *wald test for each variable
	test 1.newAGEcat 2.newAGEcat 3.newAGEcat 4.newAGEcat 5.newAGEcat 6.newAGEcat 7.newAGEcat 8.newAGEcat     
	
	  logit hbp i.newAGEcat overweight
		 logit hbp 
		\end{verbatim}
        
    \item b: Obtain the Pearson’s and Hosmer-Lemeshow goodness-of-fit statistics for each model and compare.
      For the mean centered age continuous variable, and overweight model, the Pearson's GOF test, has N=2325, with 94 covariate patterns, and Chi-square test statistic of 136.59, with $p=0.0014$.  We reject the null that there is a goodness of fit, and in this data suggests there are large residuals in the model.  For the Hosmer-Lemeshow, we chose a group size of 10, and saw no observations with counts less than 5.  The Hosmer-Lemeshow statistic was 13.43 and p=0.0978. These results to not agree, and have marginal residuals deviations in the age continuous model.
     \begin{verbatim}
     *GOF pearson
      logit hbp meanAGE overweight
  estat gof
  estat gof, group(10) table 
     glm hbp meanAGE overweight, family(binomial) link(logit)
     \end{verbatim}
    
    
     Using the 8 age cut categories, the Pearson's Chi-square test of GOF, has 18 covariate patterns, with Chi-statistic of 9.16, and p=0.3287. We conclude there are not significant deviations from expected.  The Hosmer-Lemeshow test with 10 groups had sufficient observations per group, and a Chi-statistic of 6.15 with p=0.6307.  Both tests agree using age as a categorical variable suggesting that there are not significant deviations from expected, and fail to reject the null.
    
     \begin{verbatim}
  ** AIC = 0.8902303
   *BIC=-15935.13
  **Pearson's: Chi=18.92, P=0.2172
  logit hbp i.newAGEcat overweight
  estat gof
  estat gof, group(10) table 
      lowess hbp newAGEcat,logit
      \end{verbatim}
    
    \item c:  Obtain the AIC and BIC for each model and compare.
     For age mean centered and continuous logistic model, the AIC =0.8866104, and BIC= -15943.56
    
     The age categorical model, logisitic regression model, has AIC=0.891796, and BIC=-15891.24.
     Both models have similar Akaike's Information criterion, and Bayesian information criterion, however for mean age centered continuous model, has smaller AIC, but larger BIC.  
     \begin{verbatim}
     glm hbp meanAGE overweight, family(binomail) link(logit)
    glm hbp i.newAGEcat overweight,family(binomial) link(logit)
     \end{verbatim}
     
    \item d:Which model do you prefer and why?  
      I prefer the model defined as $hbp \sim ageCategorical + overweight$ because the goodness of fit tests both fail to reject the null stating that there are not significant deviations from expected.  The model using age (mean centered) has a marginal goodness of fit, suggesting that there could be outliers in the data.  Both AIC/BIC statistics are comparable.  However upon examining the LOWESS curve of mean-age continuous plots of linearity, we see that there is a not a great linear fit using age as a continuous variable.
      Examining the LOWESS plot of the 8 age categories, the linearity assumptions are greatly improved. The LOWESS plot indicates a much smoother linearity assumptions, and this corresponds with the GOF test results. 
    
\end{itemize}

\section{Question 4}
 For your preferred model above, plot the diagnostic statistics and interpret. 
 
  \begin{itemize}
      \item a: Are there any problematic observations?
      For the mean age continuous variable we found 129 observations that had elta beta (cook's distance) greater than 0.25, or delta chi-square (influential points) higher than 6.  In order to improve the model, I dropped the 129 points. 
      \item b: If you did find possibly problematic observations, describe what your next steps in model fitting might be.
       Dropping the 129 observations I refit the age centered variable, and checked the goodness of fit as well as the LOWESS plot.  the GOF statistics failed to reject the null after dropping 129 observations, and the LOWESS passed visual inspection.
  \end{itemize}

 \begin{verbatim}
      logit hbp meanAGE overweight
  predict dx2, dx2
  predict dd, dd
  predict dbeta, dbeta
   list meanAGE  if dx2 > 6
   list meanAGE  if dd >10
   list meanAGE  if dbeta > 0.25
   
      list meanAGE  if dbeta > 0.25 & dx2>6
  drop if dbeta > 0.25 | dx2>6
   *drops 33 points
     logit hbp meanAGE overweight
	 lowess hbp meanAGE,logit
	 estat gof
  estat gof, group(10) table 
 \end{verbatim}

\end{document}