\documentclass{article}
\usepackage[utf8]{inputenc}

\usepackage{tabularx, booktabs}

% This package allows for automatic conversion of eps figures to pdf
\usepackage{graphics}
\usepackage{epstopdf} 
\graphicspath{ {.}{04/} }
%\DeclareGraphicsExtensions{.pdf,.eps,.png}



\usepackage{amsmath, amssymb}
\usepackage[margin=1in]{geometry}
\usepackage{graphicx}
\usepackage{xcolor}

% To set the enumeration method
\usepackage[shortlabels]{enumitem}

\usepackage[T1]{fontenc}
\usepackage[utf8]{inputenc}
\usepackage{lmodern}

\title{PM511 b: Group 7 Homework Assignment 4}
\author{Group 7}
\date{Due: Feb 29, 2019}

\begin{document}

% We use this command to change the font of the entire document. We can take it out if you want :)
\fontfamily{lmss}\selectfont

\maketitle

\section{Question 1}

\textbf{Preamble:} Before replying the data and code to import it follows



\begin{verbatim}[table6-17.csv]
therapy,gender,response,n
Sequential,Male,"Progressive Disease",28
Sequential,Male,"No Change",45
Sequential,Male,"Partial Remission",29
Sequential,Male,"Complete Remission",26
Sequential,Female,"Progressive Disease",4
Sequential,Female,"No Change",12
Sequential,Female,"Partial Remission",5
Sequential,Female,"Complete Remission",2
Alternating,Male,"Progressive Disease",41
Alternating,Male,"No Change",44
Alternating,Male,"Partial Remission",20
Alternating,Male,"Complete Remission",20
Alternating,Female,"Progressive Disease",12
Alternating,Female,"No Change",7
Alternating,Female,"Partial Remission",3
Alternating,Female,"Complete Remission",1
\end{verbatim}

\begin{verbatim}[stata code]
set more off
set trace off
clear all

// Reading the data in
import delimited using assignments/05/table6-17.csv, delimiter(",")

gen response2 = .
replace response2 = 1 if response ==  "Progressive Disease"
replace response2 = 2 if response ==  "No Change"
replace response2 = 3 if response ==  "Partial Remission"
replace response2 = 4 if response ==  "Complete Remission"

lab define responselbl  ///
1 "Progressive Disease" 2 "No Change" 3 "Partial Remission" ///
4 "Complete Remission"
lab val response2 responselbl
drop response
rename response2 response

gen alternating = therapy == "Alternating"
lab def alternatinglbl 1 "Yes (Alternating)" 0 "No (Sequential)"
lab val alternating alternatinglbl

gen female = gender == "Female"
lab def femalelbl 0 "No (Male)" 1 "Yes (Female)"
lab val female femalelbl

save assignments/05/table6-17.dta, replace
\end{verbatim}

Responses

\begin{enumerate}[a.]
\item Modeling the response as an unordered multinomial logistic regression

\begin{verbatim}
. mlogit response alternating female [fw = n], base(1) rrr

Iteration 0:   log likelihood = -399.98398  
Iteration 1:   log likelihood = -393.17752  
Iteration 2:   log likelihood = -393.01823  
Iteration 3:   log likelihood = -393.01778  
Iteration 4:   log likelihood = -393.01778  

Multinomial logistic regression                   Number of obs   =        299
                                                  LR chi2(6)      =      13.93
                                                  Prob > chi2     =     0.0304
Log likelihood = -393.01778                       Pseudo R2       =     0.0174

-------------------------------------------------------------------------------------
           response |        RRR   Std. Err.      z    P>|z|     [95% Conf. Interval]
--------------------+----------------------------------------------------------------
Progressive_Disease |  (base outcome)
--------------------+----------------------------------------------------------------
No_Change           |
        alternating |    .539857   .1595096    -2.09   0.037     .3025362    .9633413
             female |   .9137244   .3459634    -0.24   0.812     .4350356    1.919135
              _cons |   1.811411   .4203604     2.56   0.010     1.149437    2.854624
--------------------+----------------------------------------------------------------
Partial_Remission   |
        alternating |   .4074324   .1430643    -2.56   0.011     .2047236    .8108551
             female |   .6963934   .3322966    -0.76   0.448     .2733308    1.774274
              _cons |   1.128631   .2919527     0.47   0.640     .6797738    1.873871
--------------------+----------------------------------------------------------------
Complete_Remission  |
        alternating |   .4499522   .1654803    -2.17   0.030     .2188353    .9251567
             female |   .2785116   .1839689    -1.94   0.053     .0763112    1.016479
              _cons |   1.016546   .2730411     0.06   0.951     .6004782    1.720906
-------------------------------------------------------------------------------------
\end{verbatim}

In this case, the treatment, alternating vs sequential, has a significant effect in all three outcomes at the 5\% level (with respect to the baseline, which is, Progressive Disease), in particular, we see that 

\begin{itemize}
	\item \textbf{No Change} has a RRR of 0.54 (CI: [.30, .96], p-value = 0.04), so alternating treatment yields a higher chance of 
	\item \textbf{Partial Remission}
	\item \textbf{Complete Remission}
\end{itemize}

\item This question has two parts:

\begin{enumerate}[i.]
	\item 1
	
\begin{verbatim}
. constraint 1 [Partial_Remission]:alternating = [Complete_Remission]:alternating
. mlogit response alternating female [fw = n], base(1) constraint(1) rrr

Iteration 0:   log likelihood = -399.98398  
Iteration 1:   log likelihood = -393.21683  
Iteration 2:   log likelihood =  -393.0497  
Iteration 3:   log likelihood = -393.04918  
Iteration 4:   log likelihood = -393.04918  

Multinomial logistic regression                   Number of obs   =        299
                                                  Wald chi2(5)    =      12.39
Log likelihood = -393.04918                       Prob > chi2     =     0.0298

( 1)  [Partial_Remission]alternating - [Complete_Remission]alternating = 0
-------------------------------------------------------------------------------------
           response |        RRR   Std. Err.      z    P>|z|     [95% Conf. Interval]
--------------------+----------------------------------------------------------------
Progressive_Disease |  (base outcome)
--------------------+----------------------------------------------------------------
No_Change           |
        alternating |   .5398146   .1594993    -2.09   0.037       .30251    .9632734
             female |   .9129831   .3456741    -0.24   0.810     .4346908    1.917543
              _cons |   1.811757   .4205349     2.56   0.010     1.149539     2.85546
--------------------+----------------------------------------------------------------
Partial_Remission   |
        alternating |   .4264608   .1277961    -2.84   0.004     .2370294    .7672837
             female |    .695995   .3317346    -0.76   0.447     .2734607    1.771403
              _cons |   1.107906   .2756531     0.41   0.680     .6803292    1.804207
--------------------+----------------------------------------------------------------
Complete_Remission  |
        alternating |   .4264608   .1277961    -2.84   0.004     .2370294    .7672837
             female |   .2780198   .1837548    -1.94   0.053     .0761169    1.015477
              _cons |   1.040075   .2615435     0.16   0.876     .6353547    1.702601
-------------------------------------------------------------------------------------

. lrtest part1a

Likelihood-ratio test                                 LR chi2(1)  =      0.06
(Assumption: . nested in part1a)                      Prob > chi2 =    0.8021

\end{verbatim}
	
	In this case we say
	
	\item 2
	
\begin{verbatim}
. constraint 2 [Partial_Remission]:female = [Complete_Remission]:female
. mlogit response alternating female [fw = n], base(1) constraint(2) rrr

Iteration 0:   log likelihood = -399.98398  
Iteration 1:   log likelihood = -393.98465  
Iteration 2:   log likelihood = -393.93946  
Iteration 3:   log likelihood = -393.93945  

Multinomial logistic regression                   Number of obs   =        299
                                                  Wald chi2(5)    =      11.47
Log likelihood = -393.93945                       Prob > chi2     =     0.0429

( 1)  [Partial_Remission]female - [Complete_Remission]female = 0
-------------------------------------------------------------------------------------
           response |        RRR   Std. Err.      z    P>|z|     [95% Conf. Interval]
--------------------+----------------------------------------------------------------
Progressive_Disease |  (base outcome)
--------------------+----------------------------------------------------------------
No_Change           |
        alternating |   .5398146   .1594993    -2.09   0.037       .30251    .9632734
             female |   .9129831   .3456741    -0.24   0.810     .4346908    1.917543
              _cons |   1.811757   .4205349     2.56   0.010     1.149539     2.85546
--------------------+----------------------------------------------------------------
Partial_Remission   |
        alternating |   .4065061   .1430497    -2.56   0.011     .2039524    .8102243
             female |   .4936071   .2112769    -1.65   0.099     .2133263    1.142137
              _cons |   1.177925    .301017     0.64   0.522      .713829    1.943752
--------------------+----------------------------------------------------------------
Complete_Remission  |
        alternating |   .4506916   .1651508    -2.17   0.030      .219769    .9242564
             female |   .4936071   .2112769    -1.65   0.099     .2133263    1.142137
              _cons |   .9700556    .259583    -0.11   0.910      .574141    1.638984
-------------------------------------------------------------------------------------

. lrtest part1a

Likelihood-ratio test                                 LR chi2(1)  =      1.84
(Assumption: . nested in part1a)                      Prob > chi2 =    0.1746
\end{verbatim}

The LRT in this case tells that there is no significant model between the two specifications, which means, that the un-constrained model, where we are assuming different effects of gender on Partial vs Complete Remission, is not providing more information that the baseline is.
	
\end{enumerate}


\end{enumerate}

\section{Question 2}

\end{document}