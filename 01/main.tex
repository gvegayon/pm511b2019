\documentclass{article}
\usepackage[utf8]{inputenc}

\usepackage{amsmath, amssymb}
\usepackage[margin=1in]{geometry}
\usepackage{graphicx}

% To set the enumeration method
\usepackage[shortlabels]{enumitem}

\usepackage[T1]{fontenc}
\usepackage[utf8]{inputenc}
\usepackage{lmodern}

\title{PM511 b: Assignment 1}
\author{Group 7}
\date{January 2019}

\begin{document}

% We use this command to change the font of the entire document. We can take it out if you want :)
\fontfamily{lmss}\selectfont

\maketitle

\section*{Question 1}

Under the assumption that $\pi=\dfrac{1}{2}$, $Y \sim Bin(2, \dfrac{1}{2})$.

\textbf{a)}

Let $A$ be the event space for $Y$. Then $A=\{no \  heads, \ one \ head, \ two \ heads \}$ and $X(A)=\{0, 1, 2 \}$.

Then:


$P(0)=\dfrac{2!}{2! 0!}(\dfrac{1}{2})^0(\dfrac{1}{2})^2=\dfrac{1}{4}$

$P(1)=\dfrac{2!}{1! 1!}(\dfrac{1}{2})^1(\dfrac{1}{2})^1=2\dfrac{1}{4}=\dfrac{1}{2}$

$P(2)=\dfrac{2!}{0! 2!}(\dfrac{1}{2})^2(\dfrac{1}{2})^0=\dfrac{1}{4}$.


\textbf{b)}

$E(Y)=np=2 \cdot \dfrac{1}{2}= 1$.

$SD(Y)=\sqrt{np(1-p)}=\sqrt{2\cdot \dfrac{1}{2}(1-\dfrac{1}{2})}=\sqrt{\dfrac{1}{2}}=\dfrac{1}{\sqrt{2}}$. This is approximately .7071.

\textbf{c)}

$P(0)=\dfrac{2!}{2! 0!}(\dfrac{3}{5})^0(\dfrac{2}{5})^2=\dfrac{4}{25}=.16$

$P(1)=\dfrac{2!}{1! 1!}(\dfrac{3}{5})^1(\dfrac{2}{5})^1=2\dfrac{6}{25}=\dfrac{12}{25}=.48$

$P(2)=\dfrac{2!}{0! 2!}(\dfrac{3}{5})^2(\dfrac{2}{5})^0=\dfrac{9}{25}=.36$.


\section*{Question 2}

Answer:

\textbf{a)}

As $n1+n2+n3=3$, so $n3=3-n1-n2$.
When n1 and n2 are known, then n3 is fixed.
Thus, only n1 and n2 are independent variables, so it's two-dimensional.

\textbf{b)}

Total 10 possible observations: \par
(0, 0, 3); (0, 1, 2); (0, 2, 1); (0, 3, 0); (1, 0, 2); (1, 1, 1); (1, 2, 0); (2, 0, 1); (2, 1, 0); (3, 0, 0).

\textbf{c)}

$P(1, 2, 0)=\dfrac{3!}{1! 2! 0!}\cdot 0.25^1 \cdot 0.50^2 \cdot 0.25^0 = 0.188$.

\textbf{d)}

For n1 alone, it's a binomial distribution, Binomial(3, $\pi_1$)

\section*{Question 3}

Answer:

Before starting, we prepare the stata session for our work:

\begin{verbatim}
. set more off

. set trace off

. 
. clear all

. 
. // Setting working directory
. cd /home/vegayon/Dropbox/usc/clases/2019_PM511B/assignments/01
/home/vegayon/Dropbox/usc/clases/2019_PM511B/assignments/01
\end{verbatim}

\textbf{a)} To load the data we use the `use` command:

\begin{verbatim}
. use vitals.dta, clear

\end{verbatim}

\textbf{b)} We can generate the desired variable using the \texttt{gen} command in a single step by generating a boolean variable while restricting the operation to observations that have \texttt{sbp} different from missing:

\begin{verbatim}
. gen hibp = sbp > 140 if sbp != .
\end{verbatim}

\textbf{c)} To summarize the data by the new variable generated, we can combine the \texttt{summ} command with \texttt{bysort}:

\begin{verbatim}
. bysort hibp: summ sbp

-------------------------------------------------------------------------------
-> hibp = 0

    Variable |       Obs        Mean    Std. Dev.       Min        Max
-------------+--------------------------------------------------------
         sbp |      1880    121.3106     11.4915         87        140

-------------------------------------------------------------------------------
-> hibp = 1

    Variable |       Obs        Mean    Std. Dev.       Min        Max
-------------+--------------------------------------------------------
         sbp |       446    153.4013    11.41861        141        . 
\end{verbatim}

\textbf{d)} As shown by the \texttt{tab} command, 19\% of the observations have high blood pressure:

\begin{verbatim}
. tab hibp

       hibp |      Freq.     Percent        Cum.
------------+-----------------------------------
          0 |      1,880       80.83       80.83
          1 |        446       19.17      100.00
------------+-----------------------------------
      Total |      2,326      100.00
\end{verbatim}

\textbf{e)} After following the instructions, here is the output command (and result) from Stata:

\begin{verbatim}
. prtest hibp == .25

One-sample test of proportion                   hibp: Number of obs =     2326
------------------------------------------------------------------------------
    Variable |       Mean   Std. Err.                     [95% Conf. Interval]
-------------+----------------------------------------------------------------
        hibp |   .1917455   .0081627                       .175747     .207744
------------------------------------------------------------------------------
    p = proportion(hibp)                                          z =  -6.4883
Ho: p = 0.25

    Ha: p < 0.25                 Ha: p != 0.25                 Ha: p > 0.25
 Pr(Z < z) = 0.0000         Pr(|Z| > |z|) = 0.0000          Pr(Z > z) = 1.0000
\end{verbatim}

Responses for the subquestions:

\begin{enumerate}[i.]
    \item $Z=-6.4883$, $\mbox{p-value}< 0.001$.
    \item Yes, with a p-value close to 0, we can reject the null at various levels, including $\alpha=0.01$.
    \item \texttt{prtest hibp == .25}
\end{enumerate}

\textbf{f)} After following the instructions, here is the resulting command and output from Stata:

\begin{verbatim}
. cii 2326 446, wald

                                                         -- Binomial Wald ---
    Variable |        Obs        Mean    Std. Err.       [95% Conf. Interval]
-------------+---------------------------------------------------------------
             |       2326    .1917455    .0081627         .175747     .207744
\end{verbatim}

Responses for the subquestions:

\begin{enumerate}[i.]
    \item $[0.175, 0.207]$
    \item At the 95\% confidence, yes.
    \item \texttt{cii 2326 446, wald}
\end{enumerate}

\section*{Question 4}

Answer:

\textbf{a)}
Wald test: $Z=\dfrac{0-0.5}{\sqrt{\dfrac{0 \cdot (1-0)}{25}}} = \dfrac{-0.5}{0}$. So it's undefined.

\textbf{b)}
95\% Wald confidence interval for $\pi$:
$0 \pm 1.96 \sqrt{\dfrac{0\cdot1}{25}}=0$.
Since 95\% CI only include 0, rather than a range of value, which means we are definitely sure that the $\pi$ is 0. So it's not believable.

\textbf{c)}
Score test:
$Z=\dfrac{0-0.5}{\sqrt{\dfrac{0.5(1-0.5)}{25}}}=-5$, p<0.001 (di 2*normal(-5)).

\textbf{d)}
$\pm 1.96 = \dfrac{(0-\pi)}{\sqrt{\dfrac{\pi(1-\pi)}{25}}}$ \par
$28.84\pi^2-3.84\pi=0$ \par
So 95\% CI: 0, 0.133

\section*{Question 5}

\textbf{a)}

$H_0: \pi = \dfrac{1}{2}$

\textbf{i)}.  $H_A: \pi > \dfrac{1}{2}$

$P(X \leq 2)= P(X \geq 8)= .001 + .01 + .044 = 0.055$.
\newline

Thus, $p=0.055$. This does not meet our threshold of $\alpha = 0.05$, thus we fail to reject the null hypothesis that $\pi = \dfrac{1}{2}$, although the result does tend towards significance.

\textbf{ii)}  $H_A: \pi < \dfrac{1}{2}$

$P(X \leq 8)= .001+.01+.044+.117 +.205 +.246 +.205 +.117 +.044=.989$.

Thus, $p=0.989$ and we fail to reject the null hypothesis that $\pi=\dfrac{1}{2}$ for the subjects in the clinical trial.

\textbf{b)}


\textbf{i)}.  $H_A: \pi > \dfrac{1}{2}$

$P(X \leq 2)= P(X \geq 8)= .001 + .01 + \dfrac{.044}{2} = 0.033$.
\newline

Thus, $p = 0.033$ and we reject the null hypothesis that $\pi=\dfrac{1}{2}$ for the subjects in the clinical trial in favor of the alternative.

\textbf{ii)}  $H_A: \pi < \dfrac{1}{2}$

$P(X \leq 8)= .001+.01+.044+.117 +.205 +.246 +.205 +.117 +\dfrac{.044}{2}=.967$.

Conclusively, $p=0.967$ and we fail to reject the null hypothesis that $\pi=\dfrac{1}{2}$ for the subjects in the clinical trial.

\textbf{c)}

Put succinctly, hypothesis tests are a form of argument by contradiction. Thus, the null and alternative hypotheses must be contradictory propositions. For a one-sided test then, the null is truly that a specified value is greater/lesser or equal than the alternative. Thus, when adding the ordinary p-values for the one-sided tests, we are double counting the observed value for discrete random variables. This isn’t the case for mid p-values since we divide the observed value by half, thereby correcting the sum.

\textbf{i)}

\textit{Output from Stata}:

\begin{verbatim}
. bitesti 10 8 0.5, detail

        N   Observed k   Expected k   Assumed p   Observed p
------------------------------------------------------------
       10          8            5       0.50000      0.80000

  Pr(k >= 8)           = 0.054688  (one-sided test)
  Pr(k <= 8)           = 0.989258  (one-sided test)
  Pr(k <= 2 or k >= 8) = 0.109375  (two-sided test)

  Pr(k == 8)           = 0.043945  (observed)
  Pr(k == 3)           = 0.117188
  Pr(k == 2)           = 0.043945  (opposite extreme)

\end{verbatim}

\end{document}
